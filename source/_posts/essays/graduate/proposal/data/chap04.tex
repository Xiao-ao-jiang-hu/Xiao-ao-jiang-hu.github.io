% !TeX root = ../thuthesis-example.tex

\chapter{研究计划}

本研究总周期为16周,计划分为四个阶段有序推进。

\section{第一阶段:核心方案设计(第1-4周)}
基于已完成的理论模型,完成动态保真度优化路由算法的详细设计,包括权重调整策略的量化规则与伪代码。同时,设计量子网络层协议栈的架构、报文格式及信令流程,并规划其在qns-3平台上的具体集成方案。最后,细化仿真实验的对比场景与参数。

\section{第二阶段:系统实现与集成(第5-9周)}
在qns-3平台上编码实现所设计的协议栈,依次完成邻居发现与拓扑同步、路由计算引擎、路径建立信令等核心模块。本阶段重点在于工程实现,确保各模块功能正确并能在仿真环境中稳定协同工作。

\section{第三阶段:性能评估与分析(第10-13周)}
运行预先设计的仿真实验,在静态与动态多种网络场景下,批量收集并分析所提算法与基准算法在保真度、成功率及吞吐量等关键指标上的性能数据。通过可视化图表与量化分析,验证算法有效性。

\section{第四阶段:论文撰写与答辩准备(第14-16周)}
系统整理研究内容与实验成果,撰写毕业论文全文,重点完成设计与实现、实验分析等核心章节。经多次修改定稿后,整理所有材料并准备毕业答辩。