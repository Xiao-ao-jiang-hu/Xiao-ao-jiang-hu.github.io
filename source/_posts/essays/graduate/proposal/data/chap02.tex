% !TeX root = ../thuthesis-example.tex

\chapter{现有研究综述}

量子网络的研究涵盖两个紧密关联的层面:一是如何设计高效、可靠的网络协议以管理量子资源;二是如何借助经典计算工具对包含噪声的复杂量子系统进行高效模拟,以验证协议设计。本章将从这两个维度对现有研究进行梳理,旨在明确当前的研究范式、主流方法及其存在的局限。

\section{量子网络路由协议研究进展}
量子网络路由协议的核心任务是在满足量子物理约束的前提下,在多跳网络中确定建立端到端纠缠的路径。现有研究主要沿几个主流范式发展,并从早期理论探索逐步转向面向实际挑战的工程设计。

\subsection{基于链路状态与度量计算的路由范式}
这是目前最接近经典网络设计哲学且被广泛采用的方法。其核心思想是将量子链路的物理特性(如纠缠生成成功率$p$、贝尔态保真度$F$、存储器相干时间$T_2$)量化为一个或多个可计算的链路度量(metric)或成本(cost)。节点通过交换这些链路状态信息,构建网络拓扑的全局视图,并运用图论算法计算最优路径。

早期的奠基性工作如\cite{Ali2017Optimal}提出的量子链路状态算法(Q-LSA),系统地阐述了将链路权重与纠缠生成率相关联,并运用Dijkstra算法计算最优路径的框架。随后的研究更精细地整合了保真度衰减模型。例如,\cite{Caleffi2017Routing}提出的纠缠路由代数(ERA)框架,将链路建模为能够同时表征“纠缠对数量”和“保真度上限”的代数对象,为形式化分析多路径路由和保真度约束下的流问题提供了数学工具。

然而,该范式下的多数协议仍将保真度视为一个硬性约束条件。其典型路径选择逻辑是两阶段的:首先,从拓扑中筛选出所有预计保真度不低于应用阈值$F_{th}$的候选路径;其次,在这些合格路径中,再根据跳数、时延或总权重进行优化选择。这种“先过滤,后优化”的模式本质上是一种满足性策略,未能将“在资源约束内尽可能提升保真度”本身作为路由计算的单一且连续的优化目标。

\subsection{近期代表性路由协议详述}
近期的研究在应对量子网络独特挑战方面提出了更具体和创新的协议,尤其体现在处理非等方性路由度量(non-isotonic metric)以及保真度感知的动态多路径路由上。

\subsubsection{基于路由代数与KOP的优化路由协议}
针对量子网络路由度量(如期望吞吐量EXT和可达速率AR)固有的非等方性(non-isotonic)特性,直接应用经典最短路径算法(如Dijkstra)可能无法找到全局最优路径。\cite{Wang2025} 提出的KOP(K-Optimal Paths)算法系统性地解决了这一问题。

该工作的核心理论贡献在于引入 路由代数(Routing Algebra) 作为分析量子路由度量的数学基础。它首先证明,对于像EXT和AR这样的度量,其全序关系 $\preceq$ 不满足“等方性(isotonicity)”,即 $w(a) \preceq w(b)$ 不能保证 $w(a \oplus c) \preceq w(b \oplus c)$(其中 $\oplus$ 表示路径拼接)。为了克服这一障碍,作者定义了 最大等方性约化(Greatest Isotonic Reduction, $\preceq_{GR}$) 关系,它是一个保持等方性的偏序关系,并且是原全序关系的最大子关系。基于$\preceq_{GR}$,他们设计了D-Dijkstra(Dominant Dijkstra)算法,用于在每个节点找到并保留所有在$\preceq_{GR}$意义下的“支配路径”。

KOP算法在D-Dijkstra的基础上,结合了Yen’s K-最短路径算法的思想。其具体步骤是:1)使用D-Dijkstra找到源-目的节点间在$\preceq_{GR}$下的第一条最优路径。2)对于已找到的第k-1条路径,依次将其“根路径(root path)”与“支路(spur path)”拼接,并通过临时删除边和节点来避免生成环,再利用D-Dijkstra计算新的候选路径。3)在所有候选路径中,根据原始全序 $\preceq$ 选择最优先的作为第k条路径。该算法被严格证明能够找到前K条简单最优路径。为了降低计算复杂度,作者还提出了一个启发式变种H-KOP,在后续搜索中用标准Dijkstra替代D-Dijkstra,在多数情况下仍能保持接近最优的性能。

该协议的另一重要贡献是对EXT度量的深入分析与简化。作者给出了一个更易于计算和分析的EXT表达式(公式8),并利用它推导出$\preceq_{GR}$的显式判据,从而实现了对路由代数的具体计算。该工作首次为量子网络最优路由提供了坚实的理论基础和可实现的算法,但主要关注离线路径计算,其在线动态适应性有待进一步研究。

\subsubsection{保真度感知的多路径多体态分发协议}
在分发多体纠缠态(如GHZ态)时,如何在网络噪声(退相干、操作不完美)影响下,同时优化分发速率和最终态的保真度,是一个关键挑战。\cite{Sutcliffe2025} 提出了两种保真度感知的多路径路由协议,即多路径星型(mp-s)和多路径树型(mp-t)协议。

该协议的核心特点是动态路由和保真度驱动的路径选择。与单路径协议预先计算固定路由方案不同,多路径协议在每一轮(timeslot)开始时,根据当前网络的链路状态图 $G’$(表示哪些节点间已成功建立贝尔纠缠)来动态选择路由方案。对于mp-t协议,其目标是找到连接所有目标用户的斯坦纳树(Steiner Tree);对于mp-s协议,则是找到从中心节点到所有用户的边不相交路径集。

在选择具体路由时,协议以最大化最终GHZ态保真度的下界 $\mathcal{F}_{LB}$ 为目标。该下界被定义为路由方案$R$中所有边$e$的贝尔态保真度$\mathcal{F}_e$的乘积。由于$\mathcal{F}_e$随存储时间因退相干而衰减,$\mathcal{F}_{LB}$同时编码了路径长度(边数)和边龄(存储轮数)对保真度的影响。协议通过为边分配权重$c_e = -\log(w_e)$($w_e$为Werner参数),将保真度最大化问题转化为标准的最小代价流(mp-s)或斯坦纳树(mp-t)问题求解。

该协议还引入了存储器截止时间 $Q_c$ 作为调控速率-保真度权衡的关键参数。模拟结果表明,通过选择合适的$Q_c$,mp-t协议相比单路径树协议(sp-t)能实现高达8.3倍的速率提升和28\%的保真度提升。其性能增益主要源于:1)利用多路径冗余减少等待所有预定链路成功所需的平均轮数;2)动态路由能在每一轮选择当时保真度最高(最“年轻”)的链路组合,从而有效缓解退相干效应。

\section{量子网络模拟方法研究现状}
由于大规模量子硬件难以获取,利用经典计算机模拟量子网络的行为成为算法设计、性能评估和系统优化的关键手段。模拟方法的核心挑战在于,如何以可承受的计算资源,高效且准确地模拟包含噪声、纠缠和测量的大规模量子系统。

\subsection{主流量子网络模拟器及其方法}
目前已有多个专用量子网络模拟器,它们采用了不同的抽象层次和模拟策略来权衡精度与效率。

以NetSquid和SeQUeNCe为代表的模拟器,提供了高保真的量子物理层模型。它们通常采用密度矩阵或状态向量来精确模拟量子态,并能够模拟复杂的噪声过程和量子操作。然而,这种精确模拟的计算成本随量子比特数指数增长,限制了其可模拟的网络规模。QuISP等模拟器则采用了另一种思路,它不完整模拟量子态,而是通过追踪少量关键错误参数来近似评估量子态的质量,从而实现了大规模网络的快速模拟,但牺牲了模拟的普遍性和精确性。

这些模拟器构成了当前量子网络研究的重要工具,但它们普遍作为独立的仿真框架运行,难以与成熟的经典网络协议栈进行深度的、事件级同步的交互仿真,而这对于研究量子-经典混合网络的动态行为至关重要。

\subsection{控制流适应:一种针对LOCC协议的高效模拟方法}
在提升模拟效率方面,近期出现了一种面向特定协议类型的重要优化方法。\cite{Lin2024Control}在《Control Flow Adaption: An Efficient Simulation Method For Noisy Quantum Networks》一文中提出了“控制流适应”(Control Flow Adaption, CFA)方法,旨在高效模拟一类常见的“局域操作与经典通信”(LOCC)协议。

该方法的核心洞见在于,许多量子网络协议(如纠缠交换、蒸馏)的步骤由经典通信传来的测量结果控制。直接模拟需要跟踪所有可能的分支并进行大量的蒙特卡洛采样。CFA方法通过将经典控制流“吸收”到量子电路中,构造一个包含所有可能分支的“叠加”式张量网络进行计算。具体而言,它通过引入辅助量子寄存器来编码经典控制信息,并设计相应的适应酉操作(Adaption Unitary)来模拟经典计算函数,最后通过及时的张量收缩丢弃不必要的中间信息。

该方法的优势在于,对于适用协议,它能够在一次模拟中精确计算出所有可能分支的平均结果(如平均保真度、成功率),从而完全避免了蒙特卡洛模拟的随机性和大量重复运行的开销。同时,通过精心设计适应酉操作和张量收缩顺序,它能有效降低张量网络的复杂度,节省内存与计算时间。论文中以链式纠缠交换和嵌套纠缠蒸馏为例,展示了CFA方法如何将模拟复杂度从指数级降低至线性级。

\section{研究现状总结}
综上所述,在路由协议层面,当前研究已建立了基于链路状态的基本范式,并开始关注动态性、多路径和混合组网等实用问题,但在路由决策中如何将“保真度优化”作为首要且连续的优化目标,仍有深入探索的空间。在模拟方法层面,尽管已有多种专用模拟器,并且在模拟算法上出现了如CFA这样针对性强的高效方法,但仍缺乏一个能够无缝集成高保真量子模拟与成熟经典网络协议栈、并支持大规模混合网络研究的统一仿真平台。本研究后续工作,正是旨在上述两个方面进行有意义的探索与推进。