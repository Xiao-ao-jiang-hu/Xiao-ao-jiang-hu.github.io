% !TeX root = ../thuthesis-example.tex
\chapter{选题背景及意义}

\section{量子互联网的发展愿景与核心挑战}
量子互联网是未来信息技术的战略制高点,其核心目标是将地理上分离的量子处理器、量子存储器与量子传感器通过量子链路互联,形成一个支撑高级量子信息处理任务的分布式系统。这一网络将为量子应用提供支持,例如通过量子隐形传态和纠缠交换实现分布式量子计算,构建连接多个传感节点的高精度量子测量网络,以及实现理论上无条件安全的广域量子密钥分发。实现这些应用的基础资源是量子纠缠,其核心质量指标——保真度,直接决定了上层量子任务的最终性能。因此,如何在大规模、多跳的网络环境中,高效、可靠地分发高保真度的纠缠对,是构建实用化量子互联网面临的最根本挑战之一。

\section{量子网络路由}
在经典互联网中,网络层路由协议(如OSPF、BGP)的核心任务是寻找数据包从源到目的地的可行路径,其优化目标通常是时延、吞吐量或链路利用率。然而,直接将这一范式移植到量子网络将面临严重的不适应性。量子纠缠的建立具有概率性、易衰减性和资源消耗性等独特物理约束。具体而言,纠缠生成可能失败,生成的纠缠对保真度会因信道噪声和存储器退相干而随时间下降,而纠缠交换等中继操作则会引入额外的保真度损耗。因此,量子网络的路由决策不能仅基于拓扑连通性,还必须考虑物理层的状态信息,将链路保真度、生成成功率及存储器相干时间等量子特征作为核心的路径度量指标。

当前学术界针对量子路由的研究,大多将保真度视为一个二元化的约束条件。其典型思路是:首先为特定应用设定一个最低可接受的保真度阈值,然后在所有满足此阈值约束的候选路径中,优化如纠缠生成速率或跳数等其他指标。这种模式虽能保证基本功能,却未能充分利用量子网络的潜力。此外,对于分布式量子计算中的复杂电路执行或高精度量子传感而言,最终结果的误差率往往与所使用的纠缠资源的保真度成反比。这意味着,在满足基础阈值之上,主动寻求并建立保真度更高的纠缠连接,能够直接转化为应用层性能的提升。因此,将路由策略从“满足保真度阈值”演进为“在资源约束下最优化保真度”,是一个关键且尚未被充分探索的科学问题。

\section{混合网络仿真平台}
量子网络协议的设计与验证严重依赖于高性能的离散事件仿真。然而,现有的主流量子网络模拟器(如NetSquid, SeQUeNCe)均为独立的量子专用仿真框架。它们虽能精细模拟量子物理过程,但其内部的事件调度引擎与经典网络协议栈(如TCP/IP)完全隔离。这种隔离导致研究者无法在仿真中观察量子业务与经典流量共享底层光纤等基础设施时,由排队、缓存、拥塞控制等经典网络功能所引发的相互影响。例如,经典数据流的突发可能抢占波分复用信道资源,间接导致量子纠缠生成尝试的失败或时延增加,而这种复杂的跨层相互作用在现有分离的仿真工具中无法被有效捕捉。

在此背景下,qns-3模拟器的出现提供了一个解决方案。作为首个深度集成于工业级网络模拟框架NS-3的量子扩展,qns-3的创新性在于将量子物理层的事件调度完全纳入了NS-3的离散事件时间轴。这使得量子纠缠的生成、交换、测量等过程,能够与NS-3中已高度成熟的经典协议模型(如IPv4/v6路由、TCP流、HTTP业务)在同一时间轴上运行。这一特性使得qns-3成为研究“量子-经典”混合组网问题的理想仿真平台。然而,作为一个新兴平台,qns-3目前主要实现了物理层和基础的链路层模型,其网络层协议栈尚属空白,这极大地限制了利用该平台探索量子路由问题的能力。

\section{本课题拟解决的关键问题与研究内容}
基于以上分析,本课题旨在回应量子互联网发展中的两个关键需求:一是设计一种能够主动优化端到端纠缠保真度的网络层路由策略;二是在一个能真实反映量子-经典混合网络环境的仿真平台上实现并系统评估该策略。因此,本课题将聚焦并尝试解决以下科学及工程问题:

科学层面的核心问题是:如何将保真度从静态的路径约束条件,转化为动态的路径优化目标?具体而言,如何在路由计算中,综合考虑链路的实时保真度估计、存储资源的可用性以及路径建立的成功概率,设计出能够在网络资源竞争和物理衰减约束下,主动趋向于选择更高保真度路径的分布式算法。

工程实现层面的关键问题是:如何在一个成熟的、事件驱动的经典网络模拟框架(NS-3)内,设计与实现一个功能完整的量子网络层协议栈?这涉及量子路由控制报文格式的定义、与经典IP控制平面的协同、量子链路状态信息的分布式同步机制,以及最终在qns-3的量子-经典混合事件驱动引擎中的正确集成与调试。

为解决上述问题,本研究将包含以下并重的研究内容:首先,设计一种保真度感知的自适应路由算法,该算法在经典最短路径算法的基础上,引入对保真度梯度与资源消耗的联合评估。其次,依据NS-3及qns-3的软件架构规范,设计并实现量子网络层的协议模型,包括邻居发现、拓扑收集、路由表计算和路径建立等核心模块。最后,基于实现的协议栈,在qns-3中构建多跳量子网络仿真场景,设计对比实验,定量评估所提路由算法在提升平均保真度、保障高保真业务成功率等方面的性能优势,从而验证其有效性与实用性。本研究不仅旨在提出一种改进的路由思想,更力求通过在高保真混合仿真平台上的完整实现,为未来量子网络协议的标准化与实用化研究提供可靠的实验方法论与原型基础。